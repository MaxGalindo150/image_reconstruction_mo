Este estudio exploró el uso de optimización multiobjetivo mediante los algoritmos NSGA-II y MOEA/D para abordar los desafíos de regularización en la reconstrucción de imágenes fotoacústicas. Los resultados obtenidos demuestran que ambos algoritmos son herramientas poderosas que superan las limitaciones del método tradicional de la curva L. 

NSGA-II destacó por su capacidad para generar soluciones diversas, particularmente en regiones donde la fidelidad (\( f_1 \)) es prioritaria, mientras que MOEA/D mostró un mejor equilibrio en la exploración del frente de Pareto. Ambos algoritmos permitieron analizar los compromisos entre fidelidad, regularización y positividad, proporcionando una base sólida para seleccionar soluciones adaptadas a las restricciones específicas del problema.

Además, los análisis de correlaciones y proyecciones en 2D mostraron cómo interactúan los objetivos, proporcionando una mejor comprensión de las dependencias entre ellos. Esto resulta crucial para guiar futuras aplicaciones en escenarios experimentales realistas, donde el ruido y otras limitaciones físicas son factores determinantes.

En general, este trabajo sienta las bases para futuras investigaciones en la optimización de hiperparámetros y la extensión de los métodos de reconstrucción a dimensiones más altas, como en imágenes 2D y 3D. Estos avances permitirán desarrollar herramientas más robustas y adaptables para aplicaciones biomédicas y científicas.