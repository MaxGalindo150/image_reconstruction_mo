
\section{Interpretación del Frente de Pareto} \label{sec:analysis:pareto}

Los frentes de Pareto generados por NSGA-II y MOEA/D ofrecen perspectivas valiosas sobre los compromisos entre los objetivos de fidelidad (\( f_1 \)), regularización (\( f_2 \)) y positividad (\( f_3 \)). 

Para NSGA-II:
\begin{itemize}
    \item Las soluciones cercanas a la esquina inferior izquierda representan configuraciones óptimas en fidelidad y regularización, pero con una posible disminución en la positividad de las soluciones.
    \item En contraste, las soluciones hacia la parte superior derecha del frente implican un énfasis excesivo en la regularización, lo que compromete la fidelidad.
\end{itemize}

Para MOEA/D:
\begin{itemize}
    \item Las soluciones están distribuidas de manera más uniforme en el frente de Pareto, con compromisos mejor equilibrados entre los tres objetivos.
    \item Esto sugiere que MOEA/D es más efectivo para garantizar que todas las regiones del frente sean exploradas, lo que es especialmente útil cuando los objetivos son igualmente importantes.
\end{itemize}

El uso de estos algoritmos supera significativamente la limitación de la curva L, que solo identifica un único punto en el espacio de soluciones.

\section{Correlaciones entre Objetivos} \label{sec:analysis:correlations}

El análisis de correlaciones entre los objetivos muestra interacciones clave que influyen en el desempeño de las soluciones obtenidas. La Figura \ref{fig:correlation_matrix} para NSGA-II destaca:
\begin{itemize}
    \item Una correlación negativa moderada entre \( f_1 \) (fidelidad) y \( f_3 \) (positividad), lo que indica que la mejora en fidelidad tiende a comprometer la positividad.
    \item Una correlación positiva significativa entre \( f_2 \) (regularización) y \( f_3 \), sugiriendo que las soluciones con mayor regularización tienden a reducir la negatividad.
\end{itemize}

Para MOEA/D, las correlaciones muestran:
\begin{itemize}
    \item Una relación más definida entre \( f_1 \) y \( f_2 \), lo que refleja que este algoritmo gestiona de manera más estructurada los compromisos entre fidelidad y regularización.
    \item Una menor variabilidad en las correlaciones, lo que indica una exploración más balanceada del espacio de objetivos.
\end{itemize}

Estos resultados subrayan la importancia de comprender las correlaciones para orientar la selección de soluciones dependiendo de los requisitos específicos de la aplicación.

\section{Impacto del Nivel de Ruido} \label{sec:analysis:noise}

El nivel de ruido en los datos (\( \sigma^2 \)) influye significativamente en el desempeño de NSGA-II y MOEA/D, como se ilustra en las Figuras \ref{fig:rmse_nsga2} y \ref{fig:rmse_moead}:
\begin{itemize}
    \item NSGA-II mantiene un RMSE estable para valores de ruido moderado (\( \sigma^2 \leq 10^{-2} \)), lo que lo hace adecuado para aplicaciones con mediciones moderadamente ruidosas.
    \item MOEA/D muestra una mayor robustez para niveles de ruido bajo y medio, pero presenta una sensibilidad ligeramente mayor a niveles de ruido elevados (\( \sigma^2 > 10^{-1} \)).
\end{itemize}

Ambos algoritmos superan al método de la curva L, que es considerablemente más sensible al ruido, limitando su aplicabilidad en entornos experimentales realistas.

\section{Comparación entre NSGA-II y MOEA/D} \label{sec:analysis:comparison}

La comparación entre NSGA-II y MOEA/D resalta diferencias significativas:
\begin{itemize}
    \item NSGA-II prioriza la diversidad en el frente de Pareto, generando soluciones en regiones dominadas por \( f_1 \) (fidelidad).
    \item MOEA/D ofrece una distribución más equilibrada de soluciones, lo que facilita la selección en problemas con objetivos igualmente importantes.
    \item En términos de sensibilidad al ruido, NSGA-II es más adaptable en escenarios adversos, mientras que MOEA/D sobresale en problemas con restricciones definidas.
\end{itemize}

Esta comparación destaca que ambos algoritmos son complementarios, dependiendo de los requisitos específicos del problema.

\section{Conclusiones del Análisis} \label{sec:analysis:conclusions}

Este análisis demuestra la efectividad de NSGA-II y MOEA/D en problemas de regularización para reconstrucción de imágenes fotoacústicas. En comparación con la curva L, ambos algoritmos:
\begin{itemize}
    \item Proporcionan un rango más amplio de soluciones que reflejan compromisos bien definidos entre fidelidad, regularización y positividad.
    \item Incorporan objetivos adicionales que mejoran la calidad física y matemática de las soluciones.
    \item Ofrecen una mayor flexibilidad para analizar la sensibilidad del problema a diferentes niveles de ruido y restricciones.
\end{itemize}

\section{Trabajo Futuro} \label{sec:analysis:future}
\begin{itemize}
    \item Extender la reconstrucción a dominios 2D para evaluar la escalabilidad de los algoritmos.
    \item Optimizar los hiperparámetros de los algoritmos multiobjetivo para mejorar su desempeño en exploración y convergencia.
    \item Analizar más a fondo las interacciones entre objetivos para guiar el diseño de algoritmos específicos para aplicaciones biomédicas.
\end{itemize}
