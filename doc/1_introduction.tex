\lipsum[1-4]
\lipsum[6]

\section{Photoacoustic Imaging} \label{sec:intro}

The generation of images using the photoacoustic effect has been established as a valuable non-invasive technique for investigating the composition and structure of various materials. This phenomenon occurs when a photo-absorbing material interacts with light pulses, generating increases in temperature and mechanical pressure that propagate as photoacoustic signals. These signals are detected by ultrasonic transducers, enabling the creation of photoacoustic images, which are particularly relevant in biomedical applications such as the assessment of arterial structures, tissue oxygenation monitoring, gastrointestinal endoscopies, molecular imaging, and in vivo brain studies in experimental animals.

The reconstruction of photoacoustic images is carried out by capturing and analyzing the photoacoustic waves recorded by sensors placed on the surface of the investigated material. Among the most widely used reconstruction methods are Time Reversal, Delay and Sum, and Regularization techniques.

This study focuses on an innovative approach that combines an exploratory and multi-objective perspective with evolutionary algorithms. This methodology allows for the exploration of diverse solutions without the need to determine an optimal regularization term, and by carefully selecting objectives, it facilitates the reconciliation of different constraints for image reconstruction.

\subsection{Research Questions}\label{sec:ques}

This research will investigate \lipsum[][1]. Specifically, this research will provide a foundation for \lipsum[][2]. From this foundation, \lipsum[][3].


The research question is as follows:

\begin{quote}
    \textit{\lipsum[][4]}
\end{quote}

\lipsum[][5-6]
As such, this research has been discretised as to be addressed in the following sub-questions:

\begin{enumerate}[start=1,label={RQ\arabic*:},wide = 0pt, leftmargin = 3em]
    \item \textit{\lipsum[][1]}
    \item \textit{\lipsum[][2]}
    \item \textit{\lipsum[][3]}
    \item \textit{\lipsum[][4]}
\end{enumerate}

\lipsum[5-6]

\subsection{Contributions} \label{sec:cont}

This research provides the following contributions to knowledge: 
\begin{enumerate}[start=1,label={C\arabic*:},wide = 0pt, leftmargin = 3em]
	\item \lipsum[][1]. 
	\item \lipsum[][2]. 
	\item \lipsum[][3]. 
	\item \lipsum[][4]. 
	\item \lipsum[][5]. 
	\item \lipsum[][6]. 
\end{enumerate}


\section{Image Reconstruction} \label{sec:back}

\section{Multi-Objective Optimization} \label{sec:back}

\section{Overview} \label{sec:thes}
The format of this thesis is by publication, so I will be also attempting to publish this research as the following papers:
\begin{enumerate}[start=1,label={P\arabic*:},wide = 0pt, leftmargin = 3em]
	\item This is the full title of my first paper (RQ1; C1)
	\item This is where I would put my second paper \textit{\textbf{if I had any}} (RQ2; C2, C3)
	\item You get the idea (RQ3; C4, C5)
	\item ... (RQ4; C6)
\end{enumerate}

