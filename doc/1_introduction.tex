\lipsum[1-4]
\lipsum[6]

\section{Research Questions and Contributions} \label{sec:prop}

\subsection{Research Questions}\label{sec:ques}

This research will investigate \lipsum[][1]. Specifically, this research will provide a foundation for \lipsum[][2]. From this foundation, \lipsum[][3].


The research question is as follows:

\begin{quote}
    \textit{\lipsum[][4]}
\end{quote}

\lipsum[][5-6]
As such, this research has been discretised as to be addressed in the following sub-questions:

\begin{enumerate}[start=1,label={RQ\arabic*:},wide = 0pt, leftmargin = 3em]
    \item \textit{\lipsum[][1]}
    \item \textit{\lipsum[][2]}
    \item \textit{\lipsum[][3]}
    \item \textit{\lipsum[][4]}
\end{enumerate}

\lipsum[5-6]

\subsection{Contributions} \label{sec:cont}

This research provides the following contributions to knowledge: 
\begin{enumerate}[start=1,label={C\arabic*:},wide = 0pt, leftmargin = 3em]
	\item \lipsum[][1]. 
	\item \lipsum[][2]. 
	\item \lipsum[][3]. 
	\item \lipsum[][4]. 
	\item \lipsum[][5]. 
	\item \lipsum[][6]. 
\end{enumerate}




\section{Structure of the Thesis} \label{sec:thes}
The format of this thesis is by publication, so I will be also attempting to publish this research as the following papers:
\begin{enumerate}[start=1,label={P\arabic*:},wide = 0pt, leftmargin = 3em]
	\item This is the full title of my first paper (RQ1; C1)
	\item This is where I would put my second paper \textit{\textbf{if I had any}} (RQ2; C2, C3)
	\item You get the idea (RQ3; C4, C5)
	\item ... (RQ4; C6)
\end{enumerate}

