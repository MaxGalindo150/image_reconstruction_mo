\section{Optimización Multi-Objetivo para \( \lambda \)} \label{sec:method:multiobj}

Para abordar la estimación del parámetro \( \lambda \) de manera más robusta, este trabajo emplea algoritmos de optimización multi-objetivo avanzados, específicamente NSGA-II y MOEA/D. Ambos métodos están diseñados para explorar y optimizar soluciones en escenarios con objetivos múltiples y en competencia.

\subsubsection{NSGA-II} \label{sec:method:nsga}
NSGA-II se utiliza para generar un frente de Pareto que representa las posibles soluciones para \( \lambda \), optimizando los siguientes objetivos:
\begin{enumerate}
    \item Minimización del residuo: \( \| \mathbf{y} - \mathbf{H} \mathbf{d} \|_2^2 \),
    \item Minimización del término de regularización: \( \| \mathbf{d} \|_2^2 \),
    \item Penalización de valores negativos en \( \mathbf{d} \): \( \sum |\mathbf{d}_i| \text{ para } \mathbf{d}_i < 0 \).
\end{enumerate}

El enfoque de NSGA-II permite generar un conjunto diverso de soluciones que reflejan los compromisos entre los objetivos definidos. Cada punto del frente de Pareto representa un valor de \( \lambda \) y su correspondiente reconstrucción \( \hat{\mathbf{d}} \).

\subsubsection{MOEA/D} \label{sec:method:moead}
MOEA/D, por otro lado, adopta un enfoque basado en la descomposición del problema multi-objetivo en múltiples subproblemas de optimización escalar. Cada subproblema se resuelve utilizando una combinación lineal ponderada de los objetivos:

\begin{equation}
    f_{\text{scalar}} = \omega_1 \| \mathbf{y} - \mathbf{H} \mathbf{d} \|_2^2 + \omega_2 \| \mathbf{d} \|_2^2 + \omega_3 \sum |\mathbf{d}_i| \text{ para } \mathbf{d}_i < 0,
\end{equation}

donde \( \omega_1, \omega_2, \omega_3 \) son pesos que determinan la importancia relativa de cada objetivo.

\textbf{Ventajas de MOEA/D:}
\begin{itemize}
    \item Generación más eficiente del frente de Pareto al enfocarse en subproblemas individuales.
    \item Mayor flexibilidad para explorar regiones específicas del espacio de soluciones.
    \item Escalabilidad superior en problemas con un número elevado de objetivos.
\end{itemize}

El uso combinado de MOEA/D y NSGA-II permite evaluar las diferencias en la diversidad y calidad de las soluciones generadas, así como el tiempo computacional requerido para cada algoritmo.

\section{Comparación con el Método de la Curva L} \label{sec:method:comparison}

El método de la curva L proporciona un único valor óptimo de \( \lambda \) que equilibra la fidelidad de los datos y la regularización. Sin embargo, su naturaleza unidimensional limita su capacidad para explorar soluciones alternativas. La integración de MOEA/D y NSGA-II permite:
\begin{itemize}
    \item Identificar un conjunto diverso de soluciones con diferentes compromisos entre objetivos.
    \item Incorporar restricciones adicionales, como la positividad de \( \mathbf{d} \), para mejorar la plausibilidad física de las soluciones.
    \item Analizar cómo diferentes configuraciones de \( \lambda \) afectan la reconstrucción del perfil de absorción \( \mathbf{\mu} \).
\end{itemize}

\section{Evaluación Experimental con MOEA/D y NSGA-II} \label{sec:method:moead_eval}

Los experimentos realizados con MOEA/D y NSGA-II incluyen:
\begin{enumerate}
    \item Simulación de datos con diferentes niveles de ruido (\( \sigma^2 = 0.0, 10^{-4}, 10^{-3}, 10^{-2}, 10^{-1}, 1.0 \)), donde \( \sigma^2 \) representa la varianza del ruido agregado a las mediciones.
    \item Generación del frente de Pareto utilizando MOEA/D y NSGA-II para explorar configuraciones óptimas del parámetro \( \lambda \).
    \item Comparación de las soluciones generadas con las obtenidas mediante el método de la curva L en términos de fidelidad de los datos, estabilidad de la solución y plausibilidad física.
    \item Visualización y análisis del frente de Pareto, incluyendo la evaluación de la correlación entre los objetivos y el impacto del ruido en las soluciones generadas.
\end{enumerate}

Los niveles de \( \sigma^2 \) fueron seleccionados para simular condiciones prácticas y analizar la robustez de cada método frente a diferentes niveles de ruido. Esto permite evaluar la efectividad de MOEA/D y NSGA-II en escenarios donde el ruido afecta significativamente la calidad de las mediciones y, por ende, la reconstrucción del perfil de absorción \( \mathbf{\mu} \).

Los resultados destacan que MOEA/D tiende a generar soluciones más equilibradas y robustas en presencia de altos niveles de ruido, mientras que NSGA-II ofrece una mayor diversidad en el frente de Pareto en niveles de ruido bajos.

\section{Implementación Experimental con MOEA/D y NSGA-II} \label{sec:method:implementation}

La implementación experimental incluye:
\begin{itemize}
    \item Uso de Pymoo para implementar MOEA/D y NSGA-II.
    \item Comparación gráfica y numérica entre los perfiles de absorción \( \mathbf{\mu} \) reconstruidos con cada método.
    \item Análisis de la diversidad de las soluciones generadas, considerando métricas como el hipervolumen y la cobertura del frente de Pareto.
\end{itemize}

Este diseño experimental asegura una comparación justa y rigurosa entre MOEA/D, NSGA-II y la curva L, proporcionando una evaluación integral de los métodos multi-objetivo para la reconstrucción de imágenes fotoacústicas.
