% \section{Metodología} \label{sec:method}

\section{Generación de Datos Sintéticos} \label{sec:method:data}

Los experimentos realizados en este estudio se basan en datos sintéticos generados utilizando el Modelo de Espacio de Estados Lineales descrito en la Sección \ref{sec:lit:two}. Este enfoque permite controlar de manera precisa el nivel de ruido (\( \sigma \)) y la cantidad de muestras, proporcionando un entorno controlado para evaluar la efectividad de los métodos propuestos.

El problema subyacente de reconstrucción se formula como:
\begin{equation}
    \mathbf{y} = \mathbf{H} \mathbf{d} + \mathbf{w},
\end{equation}
donde:
\begin{itemize}
    \item \( \mathbf{y} \): Mediciones simuladas obtenidas a partir del modelo,
    \item \( \mathbf{H} \): Matriz del sistema que modela la propagación acústica,
    \item \( \mathbf{d} \): Vector que describe las propiedades de absorción,
    \item \( \mathbf{w} \): Ruido con distribución \( \mathcal{N}(0, \sigma^2) \).
\end{itemize}

El vector \( \mathbf{\mu} \), que describe el perfil de absorción, se calcula a partir de \( \mathbf{d} \) utilizando relaciones del modelo físico descritas en la Sección \ref{sec:lit:two}.

\subsection{Regularización de Tikhonov} \label{sec:method:tikhonov}

La regularización de Tikhonov se emplea para estabilizar la solución de este problema mal condicionado. La formulación matemática se expresa como:
\begin{equation}
    \hat{\mathbf{d}} = \argmin_{\mathbf{d}} \left\{ \| \mathbf{y} - \mathbf{H} \mathbf{d} \|_2^2 + \lambda \| \mathbf{d} \|_2^2 \right\},
\end{equation}
donde:
\begin{itemize}
    \item \( \| \mathbf{y} - \mathbf{H} \mathbf{d} \|_2^2 \): Término de fidelidad de los datos,
    \item \( \| \mathbf{d} \|_2^2 \): Término de regularización que penaliza soluciones inestables,
    \item \( \lambda \): Parámetro de regularización que equilibra estos dos términos.
\end{itemize}

El valor de \( \lambda \) es crucial para garantizar una solución estable y precisa. Métodos como la curva L identifican un único \( \lambda \) óptimo graficando el residuo \( \| \mathbf{y} - \mathbf{H} \hat{\mathbf{d}} \|_2 \) frente a la norma de regularización \( \| \hat{\mathbf{d}} \|_2 \). En este trabajo, NSGA-II se utiliza como una alternativa para explorar múltiples valores de \( \lambda \) simultáneamente.

\subsection{Optimización Multi-Objetivo para \( \lambda \)} \label{sec:method:nsga}

NSGA-II se emplea para optimizar directamente el parámetro \( \lambda \) considerando múltiples objetivos. Los objetivos definidos son:
\begin{enumerate}
    \item Minimización del residuo: \( \| \mathbf{y} - \mathbf{H} \mathbf{d} \|_2^2 \),
    \item Minimización del término de regularización: \( \| \mathbf{d} \|_2^2 \),
    \item Penalización de valores negativos en \( \mathbf{d} \), para garantizar soluciones físicamente significativas.
\end{enumerate}

El algoritmo genera un frente de Pareto que representa múltiples valores de \( \lambda \) y sus respectivas soluciones \( \hat{\mathbf{d}} \), proporcionando un conjunto de compromisos entre los objetivos definidos. Este enfoque permite analizar cómo diferentes valores de \( \lambda \) afectan tanto la precisión como la estabilidad de las soluciones.

\subsection{Comparación con el Método de la Curva L} \label{sec:method:comparison}

El método de la curva L se utiliza como referencia para evaluar las soluciones obtenidas mediante NSGA-II. Para ello:
\begin{itemize}
    \item Se calcula el vértice de la curva L para identificar el \( \lambda \) óptimo según este método.
    \item Se evalúan las soluciones obtenidas por NSGA-II en términos de los mismos objetivos (residuo y regularización).
    \item Se analiza la relación entre el conjunto de soluciones generadas por NSGA-II y el valor único de \( \lambda \) propuesto por la curva L.
\end{itemize}

Este análisis permite comparar la flexibilidad y diversidad de soluciones proporcionadas por NSGA-II frente al enfoque tradicional.

\subsection{Evaluación Experimental} \label{sec:method:evaluation}

Los experimentos incluyen:
\begin{enumerate}
    \item Simulación de datos con diferentes niveles de ruido (\( \sigma = 0, 0.05, 0.1 \)).
    \item Generación del frente de Pareto utilizando NSGA-II para explorar \( \lambda \).
    \item Comparación de los perfiles de absorción \( \mathbf{\mu} \) reconstruidos con los métodos de la curva L y NSGA-II.
    \item Análisis de la correlación entre los objetivos para interpretar los compromisos en el frente de Pareto.
\end{enumerate}

Se utilizaron herramientas de visualización para graficar el frente de Pareto y comparar las soluciones obtenidas por ambos métodos en términos de fidelidad de los datos, estabilidad y positividad.

\subsection{Implementación Experimental} \label{sec:method:implementation}

La implementación se realizó utilizando Python y bibliotecas avanzadas de optimización, incluyendo Pymoo. Los experimentos consistieron en:
\begin{itemize}
    \item Implementación del algoritmo NSGA-II para generar soluciones multi-objetivo.
    \item Aplicación de Tikhonov para calcular \( \hat{\mathbf{d}} \) y \( \mathbf{\mu} \) utilizando los valores de \( \lambda \) generados.
    \item Comparación gráfica y numérica entre los perfiles reconstruidos por los métodos.
\end{itemize}

Este diseño experimental asegura una evaluación rigurosa de las capacidades de NSGA-II para explorar valores de \( \lambda \) en comparación con métodos tradicionales como la curva L.
