\section{Optimización Multi-Objetivo para \( \lambda \)} \label{sec:method:multiobj}

Para abordar la estimación del parámetro \( \lambda \) de manera más robusta, este trabajo emplea algoritmos de optimización multi-objetivo avanzados, específicamente NSGA-II y MOEA/D. Ambos métodos están diseñados para explorar y optimizar soluciones en escenarios con objetivos múltiples y en competencia.

\subsubsection{NSGA-II} \label{sec:method:nsga}
NSGA-II se utiliza para generar un frente de Pareto que representa las posibles soluciones para \( \lambda \), optimizando los siguientes objetivos:
\begin{enumerate}
    \item Minimización del residuo: \( \| \mathbf{y} - \mathbf{H} \mathbf{d} \|_2^2 \),
    \item Minimización del término de regularización: \( \| \mathbf{d} \|_2^2 \),
    \item Penalización de valores negativos en \( \mathbf{d} \): \( \sum |\mathbf{d}_i| \text{ para } \mathbf{d}_i < 0 \).
\end{enumerate}

El enfoque de NSGA-II permite generar un conjunto diverso de soluciones que reflejan los compromisos entre los objetivos definidos. Cada punto del frente de Pareto representa un valor de \( \lambda \) y su correspondiente reconstrucción \( \hat{\mathbf{d}} \).

\subsubsection{MOEA/D} \label{sec:method:moead}
MOEA/D, por otro lado, adopta un enfoque basado en la descomposición del problema multi-objetivo en múltiples subproblemas de optimización escalar. Cada subproblema se resuelve utilizando una combinación lineal ponderada de los objetivos:

\begin{equation}
    f_{\text{scalar}} = \omega_1 \| \mathbf{y} - \mathbf{H} \mathbf{d} \|_2^2 + \omega_2 \| \mathbf{d} \|_2^2 + \omega_3 \sum |\mathbf{d}_i| \text{ para } \mathbf{d}_i < 0,
\end{equation}

donde \( \omega_1, \omega_2, \omega_3 \) son pesos que determinan la importancia relativa de cada objetivo.

\textbf{Ventajas de MOEA/D:}
\begin{itemize}
    \item Generación más eficiente del frente de Pareto al enfocarse en subproblemas individuales.
    \item Mayor flexibilidad para explorar regiones específicas del espacio de soluciones.
    \item Escalabilidad superior en problemas con un número elevado de objetivos.
\end{itemize}

El uso combinado de MOEA/D y NSGA-II permite evaluar las diferencias en la diversidad y calidad de las soluciones generadas, así como el tiempo computacional requerido para cada algoritmo.

\section{Comparación con el Método de la Curva L} \label{sec:method:comparison}

El método de la curva L proporciona un único valor óptimo de \( \lambda \) que equilibra la fidelidad de los datos y la regularización. Sin embargo, su naturaleza unidimensional limita su capacidad para explorar soluciones alternativas. La integración de MOEA/D y NSGA-II permite:
\begin{itemize}
    \item Identificar un conjunto diverso de soluciones con diferentes compromisos entre objetivos.
    \item Incorporar restricciones adicionales, como la positividad de \( \mathbf{d} \), para mejorar la plausibilidad física de las soluciones.
    \item Analizar cómo diferentes configuraciones de \( \lambda \) afectan la reconstrucción del perfil de absorción \( \mathbf{\mu} \).
\end{itemize}

\section{Evaluación Experimental con MOEA/D y NSGA-II} \label{sec:method:moead_eval}

Los experimentos realizados con MOEA/D y NSGA-II incluyen:
\begin{enumerate}
    \item Simulación de datos con diferentes niveles de ruido (\( \sigma^2 = 0.0, 10^{-4}, 10^{-3}, 10^{-2}, 10^{-1}, 1.0 \)), donde \( \sigma^2 \) representa la varianza del ruido agregado a las mediciones.
    \item Generación del frente de Pareto utilizando MOEA/D y NSGA-II para explorar configuraciones óptimas del parámetro \( \lambda \).
    \item Comparación de las soluciones generadas con las obtenidas mediante el método de la curva L en términos de fidelidad de los datos, estabilidad de la solución y plausibilidad física.
    \item Visualización y análisis del frente de Pareto, incluyendo la evaluación de la correlación entre los objetivos y el impacto del ruido en las soluciones generadas.
\end{enumerate}

Los niveles de \( \sigma^2 \) fueron seleccionados para simular condiciones prácticas y analizar la robustez de cada método frente a diferentes niveles de ruido. Esto permite evaluar la efectividad de MOEA/D y NSGA-II en escenarios donde el ruido afecta significativamente la calidad de las mediciones y, por ende, la reconstrucción del perfil de absorción \( \mathbf{\mu} \).

Los resultados destacan que MOEA/D tiende a generar soluciones más equilibradas y robustas en presencia de altos niveles de ruido, mientras que NSGA-II ofrece una mayor diversidad en el frente de Pareto en niveles de ruido bajos.

\section{Implementación Experimental con MOEA/D y NSGA-II} \label{sec:method:implementation}

La implementación experimental se diseñó para evaluar la efectividad de los algoritmos multiobjetivo MOEA/D y NSGA-II en la selección del parámetro de regularización \( \lambda \) en la reconstrucción de imágenes fotoacústicas. Este enfoque comparativo se estructuró en los siguientes componentes clave:

\subsection{Herramientas de Implementación}
Se utilizó la biblioteca Pymoo para implementar MOEA/D y NSGA-II debido a su flexibilidad para diseñar algoritmos evolutivos y personalizar operadores genéticos. Las características específicas incluyeron:
\begin{itemize}
    \item Implementación de MOEA/D con particiones de pesos uniformes (\textit{reference directions}) para explorar de manera equitativa todo el frente de Pareto.
    \item Configuración de NSGA-II para maximizar la diversidad del frente mediante operadores de selección basados en dominancia y mantenimiento de diversidad con \textit{crowding distance}.
    \item Personalización de operadores de cruce y mutación para mejorar la exploración del espacio de soluciones:
    \begin{itemize}
        \item \textbf{Cruce Simulado de Binarios (SBX):} Utilizado en ambos algoritmos para generar soluciones variadas respetando los límites del problema.
        \item \textbf{Mutación Polinomial:} Configurada con una probabilidad de 0.8 para asegurar una exploración constante del espacio de búsqueda.
    \end{itemize}
\end{itemize}

\subsection{Configuración de los Experimentos}
Se realizaron simulaciones utilizando datos sintéticos generados con base en el modelo de espacio de estados lineales descrito en la Sección \ref{sec:lit:two}. Las configuraciones específicas incluyeron:
\begin{itemize}
    \item Población inicial generada mediante un muestreo Latin Hypercube para asegurar una distribución uniforme de las soluciones iniciales en el espacio de diseño.
    \item Definición de tres objetivos principales para cada algoritmo:
    \begin{enumerate}
        \item Minimización del residuo \( \| \mathbf{y} - \mathbf{H} \mathbf{d} \|_2^2 \).
        \item Minimización del término de regularización \( \| \mathbf{d} \|_2^2 \).
        \item Penalización de la negatividad para garantizar soluciones físicamente significativas.
    \end{enumerate}
    \item Configuración del número de generaciones (\( n_{gen} \)) en 500 y tamaños de población (\( pop\_size \)) de 300 para ambos algoritmos.
\end{itemize}

\subsection{Métricas de Evaluación}
Para evaluar los algoritmos MOEA/D y NSGA-II, se consideraron las siguientes métricas:
\begin{itemize}
    \item \textbf{Hipervolumen:} Calculado respecto a un punto de referencia común para medir la calidad y diversidad del frente de Pareto.
    \item \textbf{Distribución en el Frente:} Análisis visual y cuantitativo de la uniformidad de las soluciones generadas.
    \item \textbf{Reconstrucción del Perfil de Absorción (\( \mathbf{\mu} \)):} Comparación gráfica de los perfiles obtenidos con los valores de \( \lambda \) seleccionados por cada algoritmo.
    \item \textbf{Comparación con la Curva L:} Evaluación de las soluciones obtenidas por ambos algoritmos en comparación con el parámetro \( \lambda \) propuesto por la curva L.
\end{itemize}

\subsection{Pipeline Experimental}
El pipeline experimental incluyó los siguientes pasos:
\begin{enumerate}
    \item \textbf{Generación de Datos:} Se simularon señales fotoacústicas con diferentes niveles de ruido (\( \sigma^2 = 0.0, 10^{-4}, 10^{-3}, 10^{-2}, 10^{-1}, 1.0 \)).
    \item \textbf{Ejecución de Algoritmos:} Se ejecutaron MOEA/D y NSGA-II para cada nivel de ruido, optimizando los tres objetivos definidos.
    \item \textbf{Reconstrucción de Imágenes:} Se seleccionaron las mejores soluciones basadas en un puntaje ponderado de los objetivos (\( w_1 = 0.6 \) para \( f_1 \), \( w_2 = 0.1 \) para \( f_2 \), y \( w_3 = 0.3 \) para \( f_3 \)).
    \item \textbf{Evaluación del Frente de Pareto:} Se analizaron las métricas de hipervolumen, distribución y correlaciones entre objetivos.
    \item \textbf{Análisis Comparativo:} Se compararon los resultados de MOEA/D y NSGA-II con la solución propuesta por la curva L, considerando fidelidad, regularización y penalización de negatividad.
\end{enumerate}

\subsection{Visualización y Análisis}
Las soluciones generadas se visualizan mediante:
\begin{itemize}
    \item Diagramas de dispersión 2D y 3D de los frentes de Pareto para interpretar los compromisos entre objetivos.
    \item Gráficas de convergencia del hipervolumen para analizar la estabilidad de los algoritmos a lo largo de las generaciones.
    \item Comparación gráfica de los perfiles de absorción reconstruidos con los valores de \( \lambda \) obtenidos.
\end{itemize}

\subsection{Conclusión de la Implementación}
Este diseño experimental asegura una evaluación justa y robusta de MOEA/D y NSGA-II en comparación con la curva L. Los resultados destacan la capacidad de ambos algoritmos para generar soluciones diversas y físicamente consistentes, proporcionando un marco flexible para la selección del parámetro de regularización en problemas mal condicionados como la reconstrucción de imágenes fotoacústicas.
