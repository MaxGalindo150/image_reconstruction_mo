\section{Photoacoustic Imaging} \label{sec:lit}

\subsection{Fundamentals of Photoacoustic Imaging Reconstruction}
% Here, introduce the basic concepts of photoacoustic imaging and its applications, followed by a description of the reconstruction process and underlying physical models.

\subsection{Linear State Space Model for Photoacoustic Imaging Reconstruction}
% This section describes how linear state space models are applied in the context of photoacoustic image reconstruction, including equations and the mathematical justification.

\section{Regression Methods} \label{sec:lit:first}

\subsection{Least Squares Regression} \label{sec:lit:first:one}
% Introduce the basic least squares method, highlighting its simplicity and limitations in high-dimensional reconstruction problems.

\subsection{Ridge Regression} \label{sec:lit:first:two}
% Discuss how ridge regression addresses overfitting issues by introducing a regularization penalty.

\subsection{Regularization} \label{sec:lit:first:three}
% Explain various regularization techniques essential for stabilizing the reconstruction problem.

\subsubsection{Lasso} \label{sec:lit:first:three:one}
% Emphasize how Lasso imposes a penalty that encourages sparsity in coefficients.

\subsubsection{Tikhonov Regularization} \label{sec:lit:first:three:two}
% Describe the Tikhonov regularization technique, a popular method for stabilizing ill-conditioned reconstruction problems.

\section{Multi-Objective Optimization for Image Reconstruction} \label{sec:lit:second}

\subsection{Introduction to Multi-Objective Optimization} \label{sec:lit:second:one}
% Describe the principles of multi-objective optimization in the context of photoacoustic image reconstruction, emphasizing the balance between accuracy and efficiency.

\subsection{NSGA-III} \label{sec:lit:second:two}
% Dive into NSGA-III and its advantages for solving multi-objective problems in this study.

\section{Limitations and Challenges in Photoacoustic Imaging Reconstruction}
% Discuss current limitations of the presented methods and algorithms, potential improvements, and future challenges in the field.

% \lipsum[1-4] % Placeholder text if additional context or descriptions are needed.
