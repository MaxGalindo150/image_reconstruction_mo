\section{Revisión de Literatura} \label{sec:lit}

\subsection{Reconstrucción de Imágenes Fotoacústicas} \label{sec:lit:one}

La imagen fotoacústica combina los principios de la óptica y la acústica para generar imágenes a partir de señales ultrasónicas. Estas señales son producidas por materiales que absorben pulsos de luz, lo que genera un aumento de temperatura y ondas de presión mecánica detectables por transductores ultrasónicos. Esta técnica es ampliamente utilizada en aplicaciones biomédicas, como la evaluación de tejidos arteriales, la monitorización de oxigenación tisular y estudios cerebrales in vivo en modelos animales \cite{Xu2015}.

El proceso de reconstrucción de imágenes implica la captura y análisis de las señales registradas por sensores. Entre los métodos clásicos se encuentran \textit{Time Reversal} y \textit{Delay and Sum}, conocidos por su rapidez pero limitados frente a ruido y problemas mal condicionados. En este contexto, la regularización de Tikhonov se ha convertido en una herramienta esencial para estabilizar los problemas mal planteados \cite{Tikhonov1963}.

\subsection{Modelo de Espacio de Estados Lineales para la Reconstrucción Fotoacústica} \label{sec:lit:two}

El problema de reconstrucción puede formularse como:
\begin{equation}
    \mathbf{y} = \mathbf{H} \mathbf{d} + \mathbf{w},
\end{equation}
donde $\mathbf{y}$ representa las mediciones, $\mathbf{H}$ es la matriz de propagación acústica, $\mathbf{d}$ el vector de absorción y $\mathbf{w}$ un término de ruido. Resolver este sistema frecuentemente requiere técnicas avanzadas debido a su sensibilidad al ruido y a la naturaleza mal condicionada de $\mathbf{H}$ \cite{Candes2006}.

\subsection{Regularización de Tikhonov y el Método de la Curva L} \label{sec:lit:tikhonov}

La regularización de Tikhonov introduce un término de penalización para estabilizar la solución:
\begin{equation}
    \hat{\mathbf{d}} = \argmin_{\mathbf{d}} \left\{ \| \mathbf{y} - \mathbf{H} \mathbf{d} \|_2^2 + \lambda \| \mathbf{d} \|_2^2 \right\},
\end{equation}
donde $\lambda$ controla el equilibrio entre fidelidad a los datos y estabilidad de la solución \cite{Tikhonov1963}.

El método de la curva L selecciona $\lambda$ graficando la norma del residuo contra la norma de regularización en escala logarítmica. El vértice de la curva representa el mejor compromiso entre ambos criterios \cite{Hansen1992}. Sin embargo, esta metodología puede ser subjetiva y computacionalmente costosa en problemas de alta dimensionalidad.

\section{Métodos de Optimización Multi-Objetivo} \label{sec:lit:second}

\subsection{Introducción a la Optimización Multi-Objetivo} \label{sec:lit:second:one}

La optimización multi-objetivo busca soluciones que equilibren múltiples criterios simultáneamente. En la reconstrucción fotoacústica, los objetivos típicos incluyen minimizar el residuo y garantizar la estabilidad de la solución, generando un frente de Pareto que representa compromisos entre estos objetivos \cite{Boyd2004}.

\subsection{NSGA-II en la Reconstrucción Fotoacústica} \label{sec:lit:second:two}

NSGA-II es un algoritmo evolutivo diseñado para problemas multi-objetivo. Este método clasifica soluciones en términos de dominancia y promueve la diversidad del frente de Pareto \cite{Deb2002}. Su capacidad para manejar múltiples objetivos simultáneamente lo hace ideal para explorar alternativas en la estimación del parámetro de regularización y mejorar la reconstrucción.

\section{Desafíos y Limitaciones} \label{sec:lit:third}

A pesar de los avances, la reconstrucción de imágenes fotoacústicas enfrenta desafíos significativos:
\begin{itemize}
    \item Sensibilidad al ruido en las mediciones.
    \item Dificultad para determinar un valor óptimo de $\lambda$ en escenarios complejos.
    \item Limitaciones computacionales en técnicas tradicionales como la curva L.
\end{itemize}

Este trabajo aborda estas limitaciones mediante el uso de NSGA-II para explorar un conjunto de soluciones que optimizan objetivos múltiples, incluyendo la positividad y la estabilidad, superando las limitaciones de métodos tradicionales como el de la curva L.


